%导言区
\documentclass[hyperref,UTF8]{ctexart}

\usepackage{ctex}
\usepackage{amsthm}
\usepackage{amsmath}
\usepackage{geometry}

\geometry{a4paper,left=2.7cm,right=2.7cm}
\title{未定}
\date{\today}

\newtheorem{theorem}{定理}[subsection]
\newtheorem{definition}{定义}[subsection]


%正文

\begin{document}

	\maketitle

	\tableofcontents
\section{集合}
	\subsection{基本概念}
	\begin{definition}
	集合\ 元素	
	
	指具有某种特定性质的具体的或抽象的对象汇集成的总体,这些对象称为集合的元素
	

	\end{definition}
	\begin{definition}{笛卡尔积}
	\[ A_1\times A_2 \times \dots \times A_n=\{(a_1,a_2,\dots,a_n)|a_i\in A_i\}\]
	\end{definition}

	\begin{definition}
		集合中元素的个数\[|A|\]
	\end{definition}


	\subsection{重点}
	\begin{theorem}
	两个集合A,B相等的充要条件:\[A=B\iff A\subset B,B\subset A\]
	\end{theorem}

\section{映射}

		\subsection{映射的定义}
		
		
	\begin{definition}
		映射
		\[\phi{:}A_1\times A_2\times \dots A_n \to D\]
		
		如果$A_1\times A_2\times \dots A_n$中的每一个元素$(a_1\times a_2\times \dots a_n)$都有$D$中唯一的元素$d$与之对应,那么称$\phi $是从$A_1\times A_2\times \dots A_n$到$D$的一个映射
	\end{definition}

%\marginpar 边注的字号还没设置
	\begin{definition}
		两个映射相等\marginpar{条件:定义域、陪域相同,同一元素对应像相同
			
			\heiti{注意:$\phi_1,\phi_2$的解析式是可以不同的}}
		\[\phi_1,\phi_2{:}A_1\times A_2\times \dots A_n \to D\]
		
		如果$A_1\times A_2\times \dots A_n$中的每一个元素$(a_1\times a_2\times \dots a_n)$都有\[\phi_1(a_1\times a_2\times \dots a_n)=\phi_1(a_1\times a_2\times \dots a_n)\]
		则称这两个映射相等
	\end{definition}

	\begin{definition}单射\marginpar{单射时当且仅当$\phi(a)=\phi(b)\Rightarrow a =b$}
		%: 用 {\ :\  } 有空格
		
	设 $ \phi{\ :\  }  A \to B$是一个映射,对于$\forall a,b\in A$,如果 \[a\neq b \Rightarrow \phi(a)\neq \phi(b)\]则称$\phi$是从$ A $到$ B $的单射
	\end{definition}
	
	\begin{definition}
		满射
		
		设 $ \phi{\ :\  }  A \to B$是一个映射,如果对于$\forall \ b\in B,\exists \  a\in A$有$b=\phi(a)$
		
		则称$\phi$是从A到B的满射
	\end{definition}

    \begin{definition}
		双射(一一映射)
		
		既是单射,又是满射的映射 
	\end{definition}

	\begin{definition}
		复合映射
		
		设$ f:A\to B $和 $ g:B \to C $是两个映射
		
		规定$ g\circ f:A\to C $为对于 $\forall \ x \in A ,g\circ f(x) =g(f(x))$
		
		则称 $ g\circ f $ 为f与g的复合映射
	\end{definition}

	\begin{definition}
		逆映射
		
		设$ f:A\to B $和 $ g:B \to A $是两个映射.
		
		如果 $ f\circ g = id_B:B\to B\  and\ g\circ f = id_A:A\to A $
		
		则称 $ f $与$ g $互为逆映射 
	\end{definition}


	\subsection{映射的定理}
\begin{theorem}
	单射的复合是单射;满射的复合是满射;双射的复合是双射
\end{theorem}
	\begin{proof}
	设 $ f:A\to B,g:B\to C,g\circ f $是$f$和$g$的复合映射
	
	\item[(1)]$f,g$ 为单射,则对于任意的 $a\neq b \in A$
	
	$\Rightarrow f(a_1)\neq  f(a_2)\in B$
	
	$\Rightarrow (g\circ f)(a_1)=g(f(a_1)) \neq g(f(a_2))=(g\circ f)(a_2)$
	\item[(2)]$f,g$ 为满射,则对于任意的 $c\in C$
	
	$\exists\  b\in B,s.t. g(b) = c$
	
	$\exists\  a\in A,s.t. f(a)= b$
	
	所以$c=g(f(a))=(g\circ f)(a)$
	
	
	\item[(3)]  同(1)(2)
	\end{proof}	
	
	
	
	\begin{theorem}
	双射存在唯一的逆映射,且这个逆映射也是双射
	\end{theorem}
	
\begin{proof}
设 $ f:A\to B $是一个双射

规定$ f^{-1}:B\to A  $为对于任意的$ y = f(x)\in B ,f^{-1}(y)=x$

由于$ f $是一个双射,可以验证$ f^{-1} $ 是一个映射

\begin{itemize}
	\item 下证$ f^{-1} $是一个双射
	
	任意取$ f(x_1)=y_1\neq y_2=f(x_2) $,由于f为单射
	
	所以$ f^{-1}(y_1)=x_1 \neq x_2 =f^{-1}(y_2) $
	
	所以$ f^{-1} $是单射
	\\
	
	任意取$ x\in A $,由于$ f $是映射
	
	所以 存在$ y\in B f^{-1}(y)=x $ 
	所以$ f^{-1} $是满射
	
	所以$ f^{-1} $是双射
	
	\item
	下证 $ f^{-1} $是 $ f $ 的逆映射

\end{itemize} 

\end{proof}
	
	
%代数运算
\section{代数运算}
	\subsection{基本概念}
	\begin{definition}
	代数运算
	
	一个从$A\times B$到$D$的映射叫做一个$A\times B$到$D$的代数运算\marginpar{两个集合的笛卡尔积到第三个集合的映射}
	
	\[\circ {\ :\ }A\times B \to D \]
	\[(a,b) \mapsto d = a\circ b. \]
	
	如果集合$A=B=D$则称$\circ$为$A$上的代数运算,或是$\circ$具有封闭性
	\end{definition}	
	\begin{definition}结合律、交换律
		
		$\circ$是集合$A$上的一个代数运算
		
		\item[(1)] 如果对于$\forall a,b,c\in A ,$都有\[(a\circ b)\circ c = a \circ (b\circ c)\]
		
		则称$\circ$适合结合律
		\item[(2)] 如果对于$\forall a,b\in A ,$都有\[a\circ b = b\circ a\]
		
		则称$\circ$适合交换律
	\end{definition}
	\begin{definition}
		消去律
		
		$\circ$是集合$A$上的一个代数运算,对于$\forall a,b,c\in A$
		
		\item[(1)] 若\[a\circ b = a \circ c \quad \Rightarrow \quad b=c\] 
		
		则称$\circ$适合左消去律
		\item[(2)]若\[b\circ a = c \circ a \quad \Rightarrow \quad b=c\]
		
		则称$\circ$适合右消去律
		\item[(3)]若$\circ$既适合左消去律又适合和右消去律,则称$\circ$适合消去律
	\end{definition}
	\begin{definition}
		分配律
		
		$\oplus,\otimes$是集合$A$上的一个代数运算,$\forall a_1,a_2,b\in A$
		
		\item[(1)] 若\[b\otimes(a_1\oplus a_2)  = (b\otimes a_1) \oplus (b\otimes a_2)\] 
		
		则称$\otimes$对于$ \oplus $适合左分配律
		\item[(2)]若\[(a_1\oplus a_2)\otimes b = (a_1\otimes b) \oplus (a_2\otimes b)\]
		
		则称$\otimes$对于$ \oplus $适合右分配律
		\item[(3)]若$\circ$既适合左分配律又适合和右分配律,则称$\otimes$对于$ \oplus $适合分配律
		
	\end{definition}

	
	
	\subsection{重点}

	

\end{document}